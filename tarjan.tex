\documentclass[12pt,oneside]{article}
\usepackage{xeCJK}
\xeCJKsetup{AutoFakeBold=true, AutoFakeSlant=true}
\setCJKmainfont{新細明體}
\setmainfont{Times New Roman}
%\setmainfont{標楷體}
%\usepackage{ctex}

\usepackage{xcolor}
\usepackage{amsmath} %數學庫
\usepackage{setspace}
\usepackage{fancyhdr} %排版用,頁數
\usepackage[colorlinks=true, linkcolor=black, citecolor=black, urlcolor=blue]{hyperref}

\usepackage{graphicx} %插入图片的宏包
\usepackage{float} %设置图片浮动位置的宏包
\usepackage{subfigure} %插入多图时用子图显示的宏包
\usepackage{chngpage}
\usepackage{emptypage}


\usepackage[margin=2cm]{geometry} %上下左右的流空
\usepackage{diagbox} %表格的斜線標題

\usepackage{spacingtricks} %表格間距

%------ c++顏色 (我不會 :poop:)) ---- %
\usepackage{listings}
\definecolor{dkgreen}{rgb}{0,0.6,0}
\definecolor{gray}{rgb}{0.5,0.5,0.5}
\definecolor{mauve}{rgb}{0.58,0,0.82}
\lstset{
	numbers=left,  
	frame=tb,
	aboveskip=3mm,
	belowskip=3mm,
	showstringspaces=false,
	columns=flexible,
	framerule=1pt,
	rulecolor=\color{gray!35},
	backgroundcolor=\color{gray!5},
	basicstyle={\ttfamily},
	numberstyle=\tiny\color{gray},
	keywordstyle=\color{blue},
	commentstyle=\color{dkgreen},
	stringstyle=\color{mauve},
	breaklines=true,
	breakatwhitespace=true,
	tabsize=3,
}
%原文链接:https://blog.csdn.net/weixin_50012998/article/details/109455978
%------ c++顏色 (我不會 :poop:)) ---- %
%--------------------color---------------------------%
\definecolor{group_color_1}{HTML}{1D4765}
\definecolor{group_color_2}{HTML}{E5e6e1}
\definecolor{group_color_3}{HTML}{A7d1d9}
\definecolor{group_color_4}{HTML}{bebbb7}

\definecolor{TreeEdge}{HTML}{1B998B}
\definecolor{BackEdge}{HTML}{4E6EE4}
\definecolor{CrossEdge}{HTML}{8544DA}
\definecolor{ForwardEdge}{HTML}{534d56}

%--------------------color---------------------------%


%畫流程圖
\usepackage{tikz, mathpazo}
\usetikzlibrary{shapes.geometric, arrows}
\usetikzlibrary{shapes, shapes.multipart}
\usetikzlibrary{calc, positioning}
\usetikzlibrary{backgrounds}
\usetikzlibrary{fit}


\renewcommand{\baselinestretch}{1.5}
\renewcommand{\contentsname}{目錄}

\title{ Tarjan }
\author{ 進階教學組 }
\date{ \today }

\begin{document}

%----------------初始化------------------%


%---------縮放------------%
\tikzset{global scale/.style={
    scale=#1,
    every node/.append style={scale=#1}
  }
}
\tikzset{Decision_shape/.style={
    trapezium, draw, trapezium left angle=60,
    trapezium right angle=-60
}
}

\tikzset{
    basic box/.style = {
        shape=rectangle,
        align=center, 
        draw=#1,
        dashed,
        thick, 
        rounded corners
    }
}

%---------縮放------------%



%--------流程圖樣式---------------%
\tikzstyle{startstop} = [rectangle, rounded corners, minimum width=3cm, minimum height=1cm,text centered,text=white, draw=black, fill=group_color_1, text width=3cm]
\tikzstyle{io} = [trapezium, trapezium left angle=70, trapezium right angle=110, minimum width=3cm, minimum height=1cm, text centered, draw=black, fill=group_color_2, text width=3cm]
\tikzstyle{process} = [rectangle, minimum width=3cm, minimum height=1cm, text centered, draw=black, fill=group_color_3, text width=3cm]
\tikzstyle{decision} = [Decision_shape, minimum width=3cm, minimum height=1cm, text centered, draw=black, fill=group_color_4, text width=3cm]
%--------流程圖樣式---------------%

%------ 圖論樣式 ------ %
\tikzstyle{node} = [circle, minimum width=0cm, minimum height=0cm, text centered, draw]
\tikzstyle{arrow} = [->, thick]
%------ 圖論樣式 ------ %


%---------設置最大節點䥗度------------%
\tikzset {
MaxWidth3/.style={
    text width=3cm
}
}
%---------設置最大節點䥗度------------%




%---------------初始化-------------------%
\pagestyle{empty}
\maketitle

\thispagestyle{empty}

\clearpage

\tableofcontents
\pagestyle{plain}
\setcounter{page}{0}
\clearpage

\section{DFS tree}
\subsection{概念}
{\color{red}這並不是真正的一個tree},
\textbf{DFS tree}是一個在有向或無向圖中,
把有成功進行到\textbf{DFS}的邊當成樹邊,
然後變成 一棵樹 + 其他邊 的形式,
所以依據選的根結點的不同、\textbf{DFS}順序不同,
\textbf{DFS tree}也會跟著改變。

在實作上,
我們也不會特別的把他求出來,
只是剛好在\textbf{DFS}的過程中他會出現而已,
而且他的特性會比較方便之後的講解
\vspace*{3em}
\begin{lstlisting}[language=C++,escapeinside=``]
func DFS_tree(node current) {
    if ( current.isvisited ) 
        return;
    
    //a -> b (edge)
    //from a to b
    //edge.from = a, edge.to = b
    for edge in edge_from_current 
        if( not edge.to.isvisited )
            edge.to.state <- DFS_tree_node
            DFS_tree( edge.to )
        else   
            edge.to.state <- other_node
}
\end{lstlisting}
\vspace*{3em}

上文的虛擬碼是找出\textbf{DFS tree}的邊,
因為實作不複雜而且概念也十分基本就不放完整程式碼。

可以發現,
圖上其實會有些邊沒有被選到,
於是為了方便分類,
人們幫這些沒有被選到的邊分了類並取了名字

\begin{enumerate}
    \item 
        樹邊(Tree Edge): 在\textbf{DFS tree}上的邊
    \item 
        回邊(Back Edge): 從\textbf{子節點}連到\textbf{父節點}的邊
    \item 
        前向邊(Forward Edge): 從\textbf{父節點}連到\textbf{子結點的子樹結點}的邊
    \item 
        {\color{blue}交錯邊}(Cross Edge) : 連到兩個非祖孫關係的邊
\end{enumerate}

換個說法,
你可以把\textbf{DFS tree}想像成一個族譜圖,
然後把邊想像成認識的關係圖(有向邊就是 你認識他,但他不認識你,可能過年的時候很常出現 :poop:),
如果以自身來比喻的話有點像這樣

\begin{enumerate}
    \item 樹邊: 你認識你的孩子
    \item 回邊: 你認識輩分比你大的直系血親
    \item 前向邊: 和回邊互補, 識輩分比你大的直系血親認識你
    \item 交錯邊: 你認識你的旁系血親
\end{enumerate}

\begin{figure}[H]
\centering
\begin{tikzpicture}[node distance=2cm][H]
\def\GlobalxShift{2cm}
    \node (1) [node] {1};
    \node (2) [node, below of=1] {2};
    \node (3) [node, below of=2, xshift={-\GlobalxShift}] {3}; 
    \node (4) [node, below of=3]{4};
    \node (5) [node, below of=4]{5};

    \node (6) [node, below of=2, xshift={\GlobalxShift}] {6}; 
    \node (7) [node, below of=6]{7};

    \node (8) [node, left of=1, xshift={-\GlobalxShift}] {7};

    \draw [TreeEdge, arrow] (1) -- (2);
    \draw [TreeEdge, arrow] (2) -- (3);
    \draw [TreeEdge, arrow] (3) -- (4);
    \draw [TreeEdge, arrow] (4) -- (5);
    \draw [TreeEdge, arrow] (2) -- (6);
    \draw [TreeEdge, arrow] (6) --node[anchor=west]{Tree} (7);

    \draw[ CrossEdge, arrow] (8) --node[anchor=west]{Cross} (3);
    \draw[ CrossEdge, arrow] (8) -- (5);

    \draw[ BackEdge, arrow] (5) --node[anchor=west]{Back} (2);

    \draw[ ForwardEdge, arrow] (1) --node[anchor=west]{Forward} (6);
\end{tikzpicture}
\end{figure}
(圖片原形by建中)




\clearpage
\section{連通}
\paragraph{定義}:把邊全部轉成無向邊之後,若結點集合任意點$A$到任意點$B$都存在一條路徑,
則我們即稱圖為連通的

\vspace*{1em}
如果以上面的比喻來說的話,
就是如果你們家族都是社交大師,
不會出現只有一邊認識的情況,
這時候對於任一個親戚可以獲得的情報網就是連通分量。

有向圖其實還有分強連通跟弱連通,
強連通是指如果$A$可以到$B$,則$B$也可以到$A$,兩個人都要彼此到達對方才算連通。
弱連通則是即使$A$可以到$B$,$B$也不一定到$A$,只要有一個人可以到達對方就算連通。

\section{連通分量}
\subsection{邊雙連通分量/橋}

如果移除一個邊可以使連通的塊增加,
我們稱這個邊為橋,
在把所有的橋都移去之後,
我們就可以得到邊雙連通分量,
為什麼他要叫邊雙連通呢?
因為如果想要再增加連通塊的話至少需要移除兩個邊。



\subsection{點雙連通分量/割點}

大致上和邊雙連通分量差不多,
只是由刪除邊變成刪除點而已,
而那個點稱為割點,
如果把所有的割點都去掉之後我們就可以地到點雙連通分量。



\clearpage
\section{Tarjan}
\subsection{無向圖}

Tarjan 的無向圖比較簡單理解一些,
因為無向圖有如下幾種性質
\vspace*{1em}
\begin{adjustwidth}{2em}{0em}
\begin{enumerate}
    \item 沒有\textbf{Cross Edge} 
    \item \textbf{Forward Edge} 和 \textbf{Back Edge}被和到了一起,
            他們會同時出現

\end{enumerate}
\end{adjustwidth}
\vspace*{1em}

如果知道了以上的性質,
我們要如何找出所有的橋呢?
首先,
我們可以發現,
\textbf{Forward Edge}對答案並不會有影響,
因為我們還是可以從\textbf{Tree Edge}走過去,
所以現在我們只需要去考慮\textbf{Tree Edge}跟\textbf{Back Edge}。
先不要一次想太多,
那我們不妨從只有\textbf{Tree Edge}開始,
之後再一個個的把\textbf{Back Edge}加回去。

對於\textbf{Tree Edge}而言,
可以發現他每一個節點都是橋,
因為只要刪掉任何一個邊都會多出一個連通塊,
此時,你會發現其實對於無向圖來說,
{\color{red} 最終有大小大於2的連通塊=有環}。
對於一個\textbf{Tree}來說,
只要增加\textbf{Back Edge}就會出現\textbf{環},
來觀察一下\textbf{環}的性質,
你可以發現,
\textbf{環}會是\textbf{Back Edge}中間的那一條練。

那要如何去維護呢,
我們可以對每個點$u$紀錄一個$low$值,
即他「從$u$或$u$的子孫最多走一次回邊(可以不走),最高可以到達的點」,
接下來講結論 : 「
對於一條\textbf{Tree Edge} $e = (a, b)$ 而言,
如果$low(a) = a$則代表$e$是橋
」。
因為把$a$移除掉之後,
其子孫無法連到上面的樹,
於是就多了一個連通塊。


實作時主要是用\textbf{DFS}+\textbf{進入戳記}。
在實作時需要知道幾個觀念

\begin{adjustwidth}{2em}{0em}
\begin{enumerate}
    \item 
        \textbf{DFS}會先把所有子孫處理完才會繼續處理當前節點
    \item 
        \textbf{DFS}中,
        在沒有遇到分岔的情況下,
        遍歷的順序總是一條條練。 \label{concept_2}
    \item 
        只有互為{\color{red} 直系親屬} 才可以使用進入戳記來判斷層樹高低,
        不然如果先遞迴到的子樹其進入戳記會比後遞迴到的子樹小。 \label{concept_3}
\end{enumerate}
\end{adjustwidth}

\vspace*{2em}
\begin{center}
    虛擬碼
\end{center}
\begin{lstlisting}[language=C++,escapeinside=``]
Time_count = 0;
Stack node_save;
func Tarjan(node current) {
    current.time_in, current.low <- Time_count
    Time_count <- Time_count+1
    node_save.push( current)

    for edge in edge_from_current 
        if( not edge.to.isvisited )
            Tarjan( edge.to )
        else   
            current.low = min( current.low, edge.low) 
    
    
        
    if( current.time_in == current.low ) 
        node temp;
        do {
            temp = node_save.top()
            bcc[ temp.id ] = current.id 
            node_save.pop()
        }
        while(temp != current);
}
\end{lstlisting}
\vspace*{2em}

\subsection{有向圖}

有向圖大部分都和無向圖差不多,
不過有向邊多了\textbf{Cross Edge},
所以\ref{concept_3}的部分可能會出現錯誤,
那我們要如何去解決呢?

十分的簡單,
在從結點$u$走到\textbf{low}函數更低的節點$v$時,
我們可以先判斷$v$在不在\textbf{Stack}內。
因為如果$v$是祖先,他一定還沒被踢出\textbf{Stack},
而且根據\ref{concept_2},
如果$v$在別的子樹且還未被踢出\textbf{Stack}外的話,
代表$[ v, LCA(u, v) )$以下的節點都已經被拜訪完了,
也就是$u, v$會有共同的連通塊

\end{document}